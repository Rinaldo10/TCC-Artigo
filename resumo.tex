\begin{abstract}
A Biologia Computacional é uma área de estudo que utiliza de ferramentas da computação, matemática e estatística para modelar e solucionar problemas da biologia. Muitos problemas permeiam esta área, um destes que nos chamou atenção é o de Rearranjo de Genomas: calcular a quantidade miníma de eventos necessários para transformar um genoma em outro. Uma ramificação deste problema restringe estes eventos a um único tipo, reversões. Porém calcular  as reversões necessárias para transformar um dado genoma em outro, a fim de estabelecer uma métrica de distância evolutiva entre eles, não é um processo trivial. Neste trabalho, nos apoiamos sobre a pesquisa de \cite{kececioglu1995exact} para a entender a base teórica de distância de reversão e sua equivalência com o problema de ordenação de genomas por reversões que será o nosso objeto de estudo. Além disso implementamos os dois algoritmos propostos por \cite{kececioglu1995exact}, sendo um algoritmo aproximado e um algoritmo exato sendo este segundo uma versão mais simplificada da descrita no trabalho de Kececioglu and Sankoff. Discutimos as técnicas utilizadas em cada um deles, suas particularidades e formulamos cenários de teste com dados simulados com o intuito de avalia-los. Ao fim apresentamos análises dos experimentos e comparativos, incluindo gráficos que expressam nossos resultados e conclusões.

% A medida que computadores e tecnologias evoluem, as possibilidades de estudos e análises em diversas áreas têm se tornado mais completas, permitindo uma compreensão maior de problemas conhecidos e a construção de soluções melhores. Uma destas áreas é a Biologia Computacional, que utiliza  ferramentas e técnicas computacionais para resolver problemas no campo da Biologia. Um problema pertinente em nossos estudos é o de \textit{Ordenação de genomas por reversões}, que parte do objetivo de medir a quantidade miníma necessária de procedimentos que fazem um organismo evoluir para outro. Essa distância evolutiva é calculada sobre cadeias de genomas e os procedimentos representam mutações nos genes. Abordagens computacionais são utilizadas para modelar este problema, onde as cadeias de genomas tomam a forma de permutações, e o problema de distância se torna um problema de ordenação destas permutações e as mutações evolutivas são restritas a reversões de sub-cadeias da permutação. 

\end{abstract}