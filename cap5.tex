\chapter{Conclusão}
\label{cap:conclusao}

    Quando iniciamos o projeto, tínhamos o objetivo de seguir os estudos em \cite{kececioglu1995exact}, estudar a base  teórica necessária para entender o problema de ordenação por reversões e modelar experimentos utilizando os algoritmos \textit{Greedy} e \textit{Branch and Bound} em cenários específicos a fim de descobrir as vantagens e desvantagens dos algoritmos e suas variações. Para os experimentos que fizemos, utilizar três modelagens de permutações foi fundamental para explorar mais possibilidades. Partindo do que fizemos foi possível notar um desempenho melhor do algoritmo \textit{Greedy} na maior parte dos cenários, o que no primeiro momento parece destoar dos resultados teóricos, mas compreensível pois existem fatores que tornam isso possível. Estes fatores estão ligados inicialmente com a maneira que operamos as remoções nas listas de candidatos do \textit{Branch and Bound}. Sabemos que nossos algoritmos limitantes não são os com melhor desempenho e isso impacta diretamente na quantidade de candidatos a serem percorrido, tornando o algoritmo exato mais lento. Dessa forma, nossa proposta inicial que foi de compreender a bagagem teórica do problema e executar experimentos que demonstrassem as soluções propostas pelo autor foi cumprida. Em nosso ultimo teste de comparação entre os algoritmos onde utilizamos PGRs e nivelamos o tamanho das entradas podemos ver resultados mais precisos por parte do algoritmo exato, tendência que cresce a medida que o tamanho da permutação aumenta assim como esperávamos. 
    
    Para trabalhos futuros visamos aumentar as variações de limitantes superior e inferior para o algoritmo exato, analisar os tempos de execução e com isso gerar testes com entradas maiores e mais robustas. Em nosso período de trabalho nos deparamos com uma ferramenta chamada Gurobi, um \textit{solver} utilizado para modelar e resolver funções lineares, estas que podem ser usadas como limitantes no algoritmo exato, assim como demonstrado por \cite{kececioglu1995exact}. Entendemos que essas funções lineares trarão um desempenho superior ao que temos hoje. Ao longo do desenvolvimento reestruturamos nossos objetivos, focando apenas em versões mais simples dos algoritmos, pois entendemos que seria necessário um tempo maior para nos aprofundarmos nas particularidades de cada um deles, para assim conseguir propor mudanças e melhorias.





